\chapter{Quickstart}

\section{Startup}
\label{sec:startup}
\index{quickstart!startup}
Run from the command line without arguments, \progname{} starts up with the full GUI. If model files are given as arguments these are all loaded in to Aten, provided they are of a compatible format. Whether a model file is readable and/or writable by \progname{} is determined by the current stock of filters available to the program (see \ref{sec:filters}). These are read from the location pointed to by \$ATENDATA and, if found, from \$HOME/.aten/filters. In addition, a user preferences file named \qte{prefs.dat} may exist in \$HOME/.aten and control many aspects of the program (in fact, the preferences file is treated as a script file -- see \ref{sec:scripts}).\\

Many command-line options exist to allow for immediate loading of models, scripts, forcefields etc., and to run the program in batch mode etc. See Section \ref{sec:cli} for a full list.\\

\section{Loading Models}
\label{sec:loadit}
\index{quickstart!loading models}

If the names of model files are supplied on the command line, all are loaded into the workspace, their file formats determined by file extensions and/or content and parsed accordingly. If none of the supplied files can be loaded, the program will exit without starting the GUI. The GUI can be suppressed for when batch command-line operation is required.\\

\progname{} uses \qte{filters} to achieve import and export of model file types, potentially providing the ability to load and save any format of model data available. A filter consists of a series of commands describing the layout of the data in a given file type, and may be thought of as a small, simplified C program typically of a few tens of lines in size. Thus, filters may be written by the user to enable the import/export of data in formats tailored to the need of the individual, or to read proprietary formats not in common usage within the community. See section \ref{sec:filterhandbook} for full details on using and writing filters. \\



All input files (the preferences file, forcefields, scripts etc.) for \progname{} are entirely free-format. Comments (lines beginning with a hash \qte{\#}) are always ignored, as are blank lines (or lines containing only delimiters) in the majority of cases. Valid delimiters between datum are spaces, commas, and tabs, although all this behaviour may be overridden by the use of double or single quotes to specify, for example, filenames or titles that contain any of the delimiters. For model loading, any or all of these rules may not apply since specific formatting may bypass any or all of these rules.\\




