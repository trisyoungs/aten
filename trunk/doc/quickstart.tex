\chapter{Quickstart}
\label{sec:quickstart}
Each user using \progname{} needs a $.aten$ directory in their home directory. Within this directory live files needed by the program, such as the user's own preferences, element colour scheme, etc. Some of these the program can do without -- others will result in errors on startup. Included in the archive is an archive named \qte{dotaten.tgz} which contains standard copies of the preferences file, spacegroup list, and file filters, and should be unpacked in the home directory (see Section \ref{sec:install}).\\


\section{Startup}
\label{sec:startup}
Several files are processed before the main program begins, in a set order:\\

\begin{description}
\item[{\it \qte{\$HOME/.aten/spgrp.dat}}\its]
	{\it XXX Should be installed rather than read from user directory}. Symmetry operations for the 230 spacegroups are read from here. If the file cannot be found the program will start as normal, but crystal packing will not be available.

\item[{\it \qte{\$HOME/.aten/filters/index}}\its]
	{\it XXX Read from installed location as well as user directory?}. A list of import/output filters to be found in the this directory are given in the \qte{index} file, one per line. If the file cannot be found, no filters are read and consequently no import or export will be possible, and so the program exits.

\item[{\it \qte{\$HOME/.aten/prefs.dat}}\its]
	User preferences controlling program behaviour, input methods, colours etc. are read last. If the file cannot be found, defaults are used and the program starts as normal. For a full list of possible options see Section \ref{sec:prefs}.

\end{description}

Run from the command line without arguments, \progname{} starts up with the full GUI. If model files are given as arguments, these are loaded after the above actions have completed. Whether a model file is readable and/or writable by \progname{} is determined by the current stock of filters available to the program (read from \qte{\$HOME/.aten/filters} and XXX). Many command-line options exist to allow for immediate loading of models, scripts, forcefields etc., and run the program in batch mode. See Section \ref{sec:cli} for a full list.\\


\section{Main Window}
Most of the main window is taken up with a canvas which is used to display and edit models and take input from the user. Each of the mouse buttons has a different action on the canvas, each of which can be set to the users taste in the preferences (see Section \ref{sec:preferences}). In addition the Shift, Ctrl, and Alt keys modify or augment these default actions. See the following sections for brief descriptions of the different modes. At the foot of the window is a message box (where output and errors are thrown) and a status bar reflecting the content of the current model displayed, listing the number of atoms and the number of selected atoms (bold value in parentheses, but only if there are selected atoms), the mass of the model, and the cell type and density of the model (if it is periodic).

{\it XXX Trajectory Frame number???}

A toolbar on the left-hand side of the window provides quick access to common functions. The first four manage models and the visible items in the main window. From the top down these are: toggle model list, add new model, delete current model, and toggle message box. The model list (hidden by default) appears on the right-hand side of the window and gives access to all the models currently loaded. Adding a new model will also add it to this list. The visibility of both the model list and the message box can be toggled with the relevant buttons giving more space to the main canvas.

\section{Changing the View}
At its most basic the canvas acts as a visualiser allowing the model to be rotated, zoomed in and out, and drawn in various different styles. By default, the right mouse button is used to rotate the molecule in the plane of the screen (right-click and hold on an empty area of the canvas and move the mouse to rotate the model) and the mouse wheel zooms in and out. Note that right-clicking on an atom brings up the atom menu (see Section \ref{sec:atommenu}). The middle mouse button translates the model in the plane of the screen -- again, click-hold and drag. Rotation and translation operate on the position and orientation of the camera and no modifications to the actual coordinates of the model are made. The view can be reset at any time from the main menu (View$\rightarrow$Reset) or by pressing Ctrl-R. Both the main menu (View$\rightarrow$Model Style) and the left-hand toolbar allow the drawing style of the model to be changed between stick, tube, sphere, scaled sphere, and \qte{individual}. The last option allows different view styles to be set for different atoms.\\

The Ctrl key changes the normal behaviour of the rotation and translation operations and forces them to be performed on the coordinates of the current atom selection instead of the camera. The centre of rotation is the geometric centre of the selected atoms.

\section{Selection}
Atom selection or picking is performed with the left mouse button -- hovering the mouse over an atom will emphasise it by drawing a circle around the edge of the atom, indicating that it can be single-clicked to highlight (select) it. Single-clicks perform \qte{exclusive} selections; that is, all other atom(s) are deselected before the clicked atom is (re)selected. Double-clicking on an atom brings up the atom list (see Section XXX). Clicking in an empty region of the canvas deselects all atoms. Clicking on an empty space in the canvas, holding, and dragging draws a rectangular selection region -- releasing the mouse button then selects all atoms whose screen coordinates are within this area. Again, this selection operation is exclusive. Inclusive selections (where already-selected atoms are not deselected) is performed by holding the Shift key. Furthermore, single-clicking on a selected atom while holding Shift will deselect the atom.\\


Out of the box the standard settings are:\\

\begin{table}[h!]
  \caption{}
  \begin{tabular}{cc|l}
\hline
	Button	& Modifier	& Action \\
\hline
	Left	& None		& Click on individual atoms to select exclusively \\
		&		& Click-hold-drag to exclusively select all atoms within rectangular region \\
		&		& Double-click to show atom list \\
		& Shift		& Click on individual atoms to toggle selection state \\
		&		& Click-hold-drag to inclusively select all atoms within rectangular region \\
\hline
	Right	& None		& Click-hold-drag to rotate camera around model \\
		&		& Click on atom to show atom menu \\
		& Ctrl		& Click-hold-drag to rotate selection in local (model) space \\
\hline
	Middle	& None		& Click-hold-drag to translate camera \\
		& Ctrl		& Click-hold-drag to translate selection in local (model) space \\
  \end{tabular}
\end{table}


\section{Load it!}
\label{sec:loadit}
If the names of model files are supplied on the command line, all are loaded into the workspace, their file formats determined by file extensions and/or content and parsed accordingly. Should the name of only one model be provided, the model list in the main view will be hidden. If none of the supplied files can be loaded, the program will exit without starting the GUI. The GUI can be suppressed for when batch command-line operation is required.\\

\progname{} uses \qte{filters} to achieve import and export of model file types, potentially providing the ability to load and save any format of model data available. A filter consists of a series of commands describing the layout of the data in a given file type, and may be thought of as a small, simplified C program typically of a few tens of lines in size. Thus, filters may be written by the user to enable the import/export of data in formats tailored to the need of the individual, or to read proprietary formats not in common usage within the community. See section \ref{sec:filterhandbook} for full details on using and writing filters. \\


%\subsection{General Considerations}
%\label{sec:genfiles}

%All input files (the preferences file, forcefields, scripts etc.) for \progname{} are entirely free-format. Comments, lines beginning with a hash (\qte{\#}), are always ignored, as are blank lines (or lines containing only delimiters) in the majority of cases. Valid delimiters between datum are spaces, commas, and tabs, although all these may be overruled by the use of double or single quotes to specify, for example, filenames or titles that contain any of the delimiters. For model loading, any or all of these rules may not apply since specific formatting may bypass any or all of these rules.




