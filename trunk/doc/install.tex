\chapter{Four Step Installation}
\label{sec:install}

\section{\emph{Disclaimer}}
\label{sec:disclaimer}

\progname{} has been compiled successfully on Linux machines running SuSE 9.3, OpenSuSE 10.2, and RedHat Enterprise 4, an Intel iMac running OS X 10.4.10, and XXX. No doubt with (or without) tweaking it will run on other systems.\\

\progname{} is in development, and as such might contain a bug or two. Should you come across a crash or an issue that needs to be fixed, \emph{please} report them so they can be fixed.\\

In addition, I can't accept responsibility for any loss of data / work / sleep / expensive computers / cats / coordinates etc., use at your own risk etc., and so on...\\

\section{Ingredients for Linux}
\label{sec:linuxinstall}
\begin{itemize}
	\item GNU Autotools 2.6
	\item A C++ compiler (e.g. g++, icc)
	\item OpenGL and glut (e.g. from freeglut)
	\item pkg-config
	\item GTK+2.0 (version 2.4 or above) or Qt4 (version 4.2 or above) if you want a GUI
	\item GtkGLExt if you want the GTK+ GUI
\end{itemize}

\section{Ingredients for Mac OS X}
\label{sec:macinstall}
\begin{itemize}
	\item GNU Autotools 2.6
	\item A C++ compiler (e.g. g++, icc)
	\item OpenGL and glut frameworks
	\item pkg-config (if using Fink or MacPorts versions of GTK+ or Qt)
	\item GTK+ (version 2.4 or above) or Qt (version 4.1 or above) if you want a GUI
\end{itemize}

When installing Qt you have several choices. Either you can get the Fink or MacPorts versions, or the native version from the TrollTech website (http://trolltech.com). Depending on which you get you might need to add in extra flags to {\sffamily configure}. The Fink and MacPorts versions provide pkg-config files to enable them to be detected and configured (and no extra {\sffamily configure} options are necessary), but the native OS X version from TrollTech does not. So, if you use the TrollTech version you must run {\sffamily configure} with the {\sffamily {-}{-}with-qt=framework} option.\\

You may find that the version of GNU Autotools you have is not recent enough. Check with {\sffamily autoconf {-}{-}version} - 2.59 is not sufficient! An up-to-date version can be installed via Fink.\\

\section{Installation}

\begin{description}
	\item[Step 1:\its] Get the source.
	\begin{itemize}
		\item Download the source distribution from the website and unpack the .tgz file, or get the latest tree from the SVN repository.
		\item Run {\sffamily ./autogen.sh} to generate the {\sffamily configure} script if you got the distribution using svn.
	\end{itemize}
	Common issues:\\
	\emph{(MAC) Running {\sffamily ./autogen.sh}, autoconf fails with \qte{configure.ac:16: error: possibly undefined macro: AC\_DEFINE}.}
	\begin{itemize}
		\item This is related to the version of pkg-config you have installed (e.g. version 0.15.1 gives this error, but version 0.21 does not) with Fink / MacPorts. Upgrade to the latest version. Incidentally, the line-number reported (16) is not the actual location of the error -- autoconf reports this wrongly (the actual error occurs around line 89 with the \qte{PKG\_CHECK\_MODULES(GTK28, ..., [AC\_DEFINE...} command).
	\end{itemize}
\end{description}

\begin{description}
	\item[Step 2:\its] Configure the build (Run {\sffamily./configure})
	\begin{itemize}
		\item Most all dependencies should be detected by the scripts provided, ensuring a pain-free build (or quick detection of problems).
		\item Select a GUI. Running {\sffamily./configure} with no options detects the operating system and builds in the Qt GUI. This can be overriden with the {\sffamily {-}{-}with-gui} switch:
		\begin{itemize}
			\item {\sffamily./configure {-}{-}with-gui=qt} builds in the Qt GUI (default).
			\item {\sffamily./configure {-}{-}with-gui=gtk} builds in the legacy GTK+ GUI.
			\item {\sffamily./configure {-}{-}with-gui=none} builds a command-line only executable.
		\end{itemize}
	\end{itemize}
	Common issues:\\
	\emph{$configure$ complains \dqte{Could not find GLUT header. Is freeglut/Mesa installed?}}
	\begin{itemize}
		\item You need some form of OpenGL implementation installed. Try the freeglut packages, and don't forget to install the freeglut-devel package as well.
	\end{itemize}
\end{description}

\begin{description}
	\item[Step 3:\its] Compile the source (Run {\sffamily make})
	\begin{itemize}
		\item Once configured successfully, run {\sffamily make} to compile the source and build the program.
		\item Go make some tea or brew some coffee.
	\end{itemize}
	Common issues:\\
	\emph{When linking I get \dqte{Undefined references to \qte{pango\_x\_font\_subfont\_xlfd}} (or similar).}
	\begin{itemize}
		\item The version of gtkglext you have installed is a bit retro (version 1.0.6 uses tokens from the pango subsystem that have since been made obsolete) so download and install version 1.2.0 from gtkglext.sourceforge.net.
	\end{itemize}

\end{description}

\begin{description}
	\item[Step 4:\its] Install (Run {\sffamily make install} or {\sffamily cp src/aten $<$destination$>$})
	\begin{itemize}
		\item After compilation is complete, you're left with the \progname{} executable in the src/ directory.
		\item Either run {\sffamily make install} to place the program in {\sffamily /usr/local/bin} if you have root priveledges, or copy the program to a local location of your choice and run it from there.
		\item The default stock of filters and spacegroup definitions is installed in the default locations if you run {\sffamily make install} (on Linux systems this is typically /usr/local/share/aten).
		\item Aten is directed to this location via the \$ATENDATA environment variable which should be set before running the program.
		\item If you haven't (or can't) run {\sffamily make install} then set \$ATENDATA to the /data directory in the source tree.
	\end{itemize}
\end{description}


