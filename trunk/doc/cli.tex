\chapter{Using as a Command Line Tool}
\label{sec:cli}
\index{command-line options}
\index{cli}


\begin{optlist}{Import}

	\item[-f $file$, {-}{-}ff $file$]
	\index{command-line options!-f}
	\index{command-line options!{-}{-}ff}
		Loads the specified forcefield file before any models specified on the command line.

	\item[{-}{-}fold, {-}{-}nofold\its]
	\index{command-line options!{-}{-}fold}
	\index{command-line options!{-}{-}nofold}
		Force or prevent initial folding of atoms to within the boundaries of the unit cell (if one is present) in loaded models, overriding filter directives.
	
	\item[{-}{-}pack, {-}{-}nopack\its]
	\index{command-line options!{-}{-}pack}
	\index{command-line options!{-}{-}nopack}
		Force or prevent generation of symmetry-equivalent atoms from spacegroup information in loaded models, overriding filter directives.
	
	\item[{-}{-}bond, {-}{-}nobond\its]
	\index{command-line options!{-}{-}bond}
	\index{command-line options!{-}{-}nobond}
		Force or prevent (re)calculation of bonding in loaded models, overriding filter directives.
	
	\item[{-}{-}centre, {-}{-}nocentre\its]
	\index{command-line options!{-}{-}centre}
	\index{command-line options!{-}{-}nocentre}
		Prevent translation of non-periodic models’ centre-of-geometry to the origin, overriding filter directives.
	
	\item[-b, {-}{-}bohr\its]
	\index{command-line options!{-}{-}bohr}
		Specifies that the unit of length used in loaded models is Bohr rather than Angstrom, and should be converted to the latter.

\end{optlist}


\begin{optlist}{Output / Debugging}

	\item[-d, {-}{-}debug\its]
	\index{command-line options!-d}
	\index{command-line options!{-}{-}debug}
		Enables debugging of subroutine calls.
	
	\item[{-}{-}debugtyping\its]
	\index{command-line options!{-}{-}debugtyping}
		Enables output from the atom typing routines.
	
	\item[{-}{-}debugfilters\its]
	\index{command-line options!{-}{-}debugfilters}
		Enables output from the filter routines.
	
	\item[{-}{-}debugall\its]
	\index{command-line options!{-}{-}debugall}
		Enables output from the most all routines.

	\item[{-}{-}debugparse\its]
	\index{command-line options!{-}{-}debugparse}
		Enables output from the file parsing routines.
		
	\item[{-}{-}verbose\its]
	\index{command-line options!{-}{-}verbose}
		Switch on verbose reporting of program actions.

\end{optlist}

\begin{optlist}{Modes}

	\item[-c ``$commands...$'', {-}{-}command ``$commands...$''\its]
	\index{command-line options!-c}
	\index{command-line options!{-}{-}command}
		The command or compound command given is executed directly after processing of all other command line options is complete, including the loading of models. Commands should be separated with semicolons.

	\item[-s $file$, {-}{-}script $file$\its]
	\index{command-line options!-s}
	\index{command-line options!{-}{-}script}
		Specifies that the script $file$ is to be loaded and run immediately following loading of all models on the command line (if any are specified). Once the script is completed, unless it is terminated by the \qte{quit} command the GUI will start (if it has not been started already), with all models (and the results of their manipulation by commands in the script) available.

	\item[-i, {-}{-}interactive\its]
	\index{command-line options!-s}
	\index{command-line options!{-}{-}script}
		Starts \progname{} in interactive mode, where script commands are typed and executed immediately. The GUI is not started by default, but may be invoked (see Section \ref{sec:scripts}).
 
\end{optlist}


XXX Other options (TODO)

-x, --convert $format$ 
-m, --map $typename=element$
